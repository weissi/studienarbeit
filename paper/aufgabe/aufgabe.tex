\documentclass[a4paper,DIV=16]{scrartcl}

\newcommand{\JWemail}[1]{\texttt{#1}}
\newcommand{\JWphone}[1]{\texttt{#1}}
\newcommand{\JWemph}[1]{\emph{#1}}
\newcommand{\JWproduct}[1]{\JWemph{#1}}
%\newcommand{\JWlone}[1]{\section{#1}}
\newcommand{\JWlone}[1]{\medskip}
\def\TReg{\textsuperscript{\textregistered}}
\def\TCop{\textsuperscript{\textcopyright}}
\def\TTra{\textsuperscript{\texttrademark}}

\usepackage[ngerman]{babel}
\usepackage{ngerman}
\usepackage[utf8]{inputenc}

%\usepackage{draftwatermark}
%\SetWatermarkText{ENTWURF}
%\SetWatermarkScale{4.0}

\setlength{\parskip}{1.3ex}
\setlength{\parindent}{1.3em}

\begin{document}

\pagestyle{empty}

\title{Anmeldung Studienarbeit zum Thema "`Multi-Core Energy Accounting"'}
\author{
  Johannes Weiß, Karlstr. 17, 76133 Karlsruhe\\
  \JWemail{uni-mcea@tux4u.de}, mobil: \JWphone{+49 179 9448602}\\
  \\
  Matrikelnummer: 1333799
}
\date{\today}
\maketitle
\thispagestyle{empty}
\pagestyle{empty}

\JWlone{Grundlagen}

Viele der aktuell verfügbaren Prozessoren besitzen sogenannte
\JWemph{performance counters}, die gewisse Ereignisse innerhalb innerhalb des
Prozessors mitzählen. Bereits in früheren Arbeiten wurde gezeigt, dass gewisse
Ereignisse des Prozessors stark mit dem momentanen Energieverbrauch des
Prozessors korrelieren. Mit einem geeigneten Energiemodell kann somit zur
Laufzeit der momentane Energieverbrauch hinreichend genau abgeschätzt werden.

\JWlone{Related Work}

Die bisherigen Arbeiten beschäftigen sich ausschließlich mit
Einprozessorsystemen, heutige Desktop- und Notebook-Computersysteme besitzen
aber in fast allen Fällen mehr als eine Recheneinheit. Wären die Recheneinheiten
einzelne, voneinander unabhängige Prozessoren, ließe sich das Energiemodell
recht einfach auf Mehrprozessorsysteme erweitern. Die aktuellen
\JWemph{multi-core} Systeme besitzen zwar mehrere Recheneinheiten -- es können
also tatsächlich mehrere Operationen zur gleichen Zeit ausgeführt werden --
allerdings sind die Recheneinheiten (alias \JWemph{cores}) nicht voneinander
unabhängig: Sie teilen sich Caches, Speicher und vieles mehr.


\JWlone{Problem}

Somit muss überprüft werden ob, und gegebenenfalls wie ein Energiemodell für
sogenannte \JWemph{multi-core Syteme} aufgestellt werden kann.  Das grundlegende
Problem wird sein, die Auswirkungen eines Ereignisses sowohl lokal pro Core als
auch global für das System sein können. Das zu entwickelnde Energiemodell muss
dem also Rechnung tragen.


\JWlone{Ziele}

Das Ziel der Studienarbeit ist es ein belastbares Energiemodell für einen
\JWproduct{Intel\TReg Core\TTra i7-2600 Processor} zu entwerfen, wobei folgende
Prozessor-Features abgeschalten bleiben: \JWemph{hyper-threading},
\JWemph{DVFS}, \JWemph{turbo mode}. Es soll später möglich sein, den momentanen
Energieverbrauch jedes laufenden Prozesses im System zu erfahren. Zudem soll das
Energiemodell auch im Mehrprogrammbetrieb funkionieren, das heißt die Summe der
Energiemengen aller Prozesse im System soll der des Gesamtsystems entsprechen.
Desweiteren sollen Werkzeuge entwickelt werden, mit denen die Entwicklung eines
Energiemodells für ähnliche Prozessoren möglichst einfach ist.


\JWlone{Lösungsansatz}

Um das Ziel zu erreichen, soll zuerst nur ein Core aktiviert werden und ein
traditionelles Energiemodell, wie bei den vorhergenden Arbeiten, aufgestellt
werden. Dazu wird eine Benchmark-Suite entwickelt, die konzentriert und
möglichst voneinander isoliert einige energierelevante Ereignisse im Prozessor
hervorruft. Aus den Daten dieser Benchmark-Suite soll dann mittels
Statistik-Software ein Energiemodell gewonnen werden. Sobald ein tragfähiges
Modell für den Einprozessor-Fall gefunden worden ist, soll untersucht werden,
was getan werden muss, um es auf ein Mehrkern-System zu erweitern.

\end{document}
