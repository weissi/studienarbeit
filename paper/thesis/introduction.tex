\JWlone{Introduction}

What is the problem?

Energy is a crucial good especially for mobile devices. Since the energy
available is either limited -- on mobile devices, or can become expensive --
on devices connected to the regular power network, we should use as little as
possible. And since we want to evaluate a whole process, the goal is not to
minimize the instantaneous power but to maximize energy-efficiency.

Even though most modern operating systems try to maximize the CPU utilization
by lowering the frequency \cite{snowdon2010operating} to eventually maximize the
energy-efficiency, this reaction is not always appropriate. A lower CPU
frequency is not always the more energy-efficient
\cite{weissel2002process,snowdon2010operating}.

Thus, for being able to develop good power management strategies, a good
power/energy estimation is critical. The approach used in this study thesis is
to use the CPU's \JWemph{performance counters}.  Since
\cite{bellosa2000benefits} there have been many papers
(\cite{Bertran2010,bertran2010decomposable,kellner03tempcontrol,isci2003runtime,
weissel2002process}, ...) doing energy estimation using performance counters
with really good results.

\begin{itemize}

\item Todays CPUs have many cores and more processes than CPU

\end{itemize}

Why Energy Accounting?

\begin{itemize}

\item Thermal Management

\item Increase Efficiency

\item Basis of Migration Decision (Merkel!!!, VMs)

\item Cost Model for Customers (die Spanier)

\end{itemize}


%#  ACKNOWLEDGEMENTS  ##########################################################
\JWltwo{Acknowledgements}
\begin{itemize}

\item Prof. Dr. Bellosa

\item Simon Kellner

\item Herr Dosch

\item James McCuller

\end{itemize}

% vim: set spell spelllang=en_us fileencoding=utf8 :
