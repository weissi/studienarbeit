\JWlone{Introduction}

Energy is a crucial resource especially for mobile devices. Since the energy
available is either limited -- on mobile devices, or can become expensive -- on
devices connected to the regular power grid, we should use as little as
possible. Even though most modern operating systems try to maximize the CPU
utilization by lowering the frequency \cite{snowdon2010operating} to eventually
maximize the energy-efficiency, this reaction is not always appropriate. A lower
CPU frequency is not always the more energy-efficient
\cite{weissel2002process,snowdon2010operating}.

To intelligently minimize the usage, an accounting of the energy for each task
of its own is the basis. For every running process, the operating system should
be aware of the present contribution to the machine's total power consumption.
Additionally, for accounting purposes the overall energy consumed by a process
should be known after its termination.

Thus, for being able to develop good power management strategies, a good live
power/energy estimation is critical. The approach used in this study thesis is
to use the CPU's \JWemph{performance monitoring counters}.  Since
\cite{bellosa2000benefits} there have been many papers
(\cite{Bertran2010,bertran2010decomposable,kellner03tempcontrol,isci2003runtime,
weissel2002process}, ...) doing energy estimation using performance counters
with really good results.

Besides these points, various other applications of energy estimation are
possible. Temperature control \cite{kellner03tempcontrol} and thermal management
\cite{merkel05tmsmpsys} are just one auxiliary field. Accounted energy can also
serve as a customer cost-model for computing centers \cite{Bertran2010}, because
it reflects, much more accurately the real costs than simple computing time
based models. Another possible application in that field are migration decisions
of either virtual machines between real machines in clusters or processes
between processing units \cite{merkel10rcscheduling}. Power-aware scheduling can
also help to improve the efficiency.


%-  charges and restrictions  --------------------------------------------------
\JWltwo{Charges and Restrictions}
\label{sec:restrictions}

Because of the limited amount of time and the start from scratch of this thesis,
some energy-relevant processor and operating system features have been disabled.
Since all of these either should work with the current model or individually
raise some kind of events, this is not seen as a major drawback of this work.
Future work may certainly be able to flawlessly integrate them into the known
model. In particular the feature which lasted disabled were:

\begin{itemize}

\item Dynamic Frequency and Voltage Scaling \cite{wiki:DVFS}

\item Hyper-Threading \cite{wiki:HT}

\item ACPI Processor States other than C0 \cite{wiki:ACPI}

\item Intel\TReg{} Turbo Boost Technology 2.0 \cite{wiki:IntelTurboBoost}

\end{itemize}

In addition to the advanced processor and operating system features meantioned
above, all auxiliary processing units could not be taken into account. The
problem with the auxiliary processing units such as the floating point unit
(FPU), MMX\cite{wiki:MMX}, SSE\cite{wiki:SSE} and AVX\cite{wiki:AVX} is that
they are not or not well covered by the CPU's performance events (see
\cite{intel2011events}). So, they act somehow as a black box not reveiling the
work they do internally. That makes it virtually impossible to count their
energy with the current model.


%#  ACKNOWLEDGEMENTS  ##########################################################
\JWltwo{Acknowledgements}
\begin{itemize}

\item Prof. Dr. Bellosa

\item Simon Kellner

\item Herr Dosch

\item James McCuller

\end{itemize}

% vim: set spell spelllang=en_us fileencoding=utf8 :
