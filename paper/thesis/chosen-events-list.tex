\begin{itemize}

\item \JWctr{CPU\_\-CLK\_\-UNHALTED}
\begin{quotation}
This is an architectural event that counts the number of thread cycles while the
thread is not in a halt state. The thread enters the halt state when it is
running the HLT instruction. The core frequency may change from time to time due
to power or thermal throttling. For this reason, this event may have a changing
ratio with regards to wall clock time.
\end{quotation}

\item \JWctr{INST\_\-RETIRED}
\begin{quotation}

This event counts the number of instructions retired from execution. For
instructions that consist of multiple micro-ops, this event counts the
retirement of the last micro-op of the instruction. Counting continues during
hardware interrupts, traps, and inside interrupt handlers. Notes:
\JWctr{INST\_\-RETIRED.\-ANY} is counted by a designated fixed counter, leaving
the four (eight when Hyper\-threading is disabled) programmable counters
available for other events. \JWctr{INST\_\-RE\-TI\-RED.\-ANY\_\-P} is counted by
a programmable counter and it is an architectural performance event. Counting:
Faulting executions of GETSEC / VM entry / VM Exit / MWait will not count as
retired instructions.

\end{quotation}

\item \JWctr{BR\_\-INST\_\-RETIRED:FAR\_\-BRANCH}
\begin{quotation}
This is a non-precise version (that is, does not use PEBS) of the event that
counts far branch instructions retired.
\end{quotation}

\item \JWctr{DSB2MITE\_\-SWITCHES}
\begin{quotation}
This event counts the number of the Decode Stream Buffer (DSB)-\-to-\-MITE
swit\-ches including all misses because of missing Decode Stream Buffer (DSB)
cache and u-arch forced misses. Note: Invoking MITE requires two or three cycles
delay.
\end{quotation}

\item \JWctr{DSB\_\-FILL:ALL\_\-CANCEL}
\begin{quotation}
This event counts the number of times when a valid Decode Stream Buffer (DSB)
fill has been actually cancelled not because of exceeding the way limit.
Cancelling Decode Stream Buffer (DSB) fill may also result, for example, from
Decode Stream Buffer Queue (DSBQ) snoop hits. This is because the Decode Stream
Buffer (DSB) full hit is guaranteed to delivery from Decode Stream Buffer (DSB).
In the B step a four-bit counter will count the number of cancel operations and
will reverse the priority upon look ing up the same set.
\end{quotation}

\item \JWctr{ILD\_\-STALL:IQ\_\-FULL}
\begin{quotation}
This event counts stall cycles when instructions cannot be written because IQ is
full. Note: If there is no Resource Allocation Table (RAT) stalls, it indicates
the decoders issue.
\end{quotation}

\item \JWctr{L2\_\-RQSTS:PF\_\-HIT}
\begin{quotation}
This event counts the number of requests from the L2 hardware prefetchers that
hit L2 cache. LLC prefetch new types
\end{quotation}

\item \JWctr{LD\_\-BLOCKS:ALL\_\-BLOCK}
\begin{quotation}
Number of cases where any load ends up with a valid block-code written to the
load buffer (including blocks due to Memory Order Buffer (MOB), Data Cache Unit
(DCU), TLB, but load has no DCU miss)
\end{quotation}

\item \JWctr{LD\_\-BLOCKS:DATA\_\-UNKNOWN}
\begin{quotation}
This event counts the number of load operations delayed due to store buffer
blocks, preceding store operations with known addresses but unknown data.
Counting happens according to the final blocking codes. This does not include
inline wakeups.
\end{quotation}

\item \JWctr{UOPS\_\-DISPATCHED:STALL\_\-CYCLES}
\begin{quotation}
This event counts cycles during which no uops were dispatched from the
Reservation Station (RS) per thread.
\end{quotation}

\end{itemize}
