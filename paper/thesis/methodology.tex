\JWlone{Design}

How did I solve the problem?

\begin{itemize}

\item measuring setup sketch

\end{itemize}

\JWltwo{Calculation of the Eletrical Work}
\label{sec:calc-work}

From elementary physics

\begin{eqnarray}
     U_R & = & R * I \\
  \iff I & = & \frac{U_R}{R}
\end{eqnarray}

and

\begin{equation}
  U_{CPU} + U_{R} = 12 V
\end{equation}

we obtain the instantaneous power of the CPU by measuring the voltage drop
across the (measuring) resistor:

\begin{eqnarray}
P_{CPU}(t) & = & (12V - U_R(t)) * \frac{U_R(t)}{R} \\
           & = & \frac{12V * U_R - {U_R}^2}{R} \\
           & \stackrel{0 < U_R \ll 1}{\approx} & \frac{12V * U_R}{R}.
\end{eqnarray}

Hence, integrating will result in the electrical work

\begin{equation}
  W = \int P_{CPU}(t)dt.
\end{equation}


%-------------------------------------------------------------------------------
\JWltwo{Measuring Setup}
\label{sec:measuring-setup}

\JWlthree{Measuring Device}

For measuring the voltage drops we chose
\JWproduct{http://sine.ni.com/nips/cds/view/p/lang/en/nid/203484}{NI USB-6218}
from \JWenterprise{http://www.ni.com}{National Instruments} because it supports
high sampling rates up to 250000 samples per second
(\SI{250}{\kilo\samples\per\second}) and is very
accurate (accuracy $< \SI{2.69}{\milli\volt}$)\cite{NISpec2009}.


%-------------------------------------------------------------------------------
\JWltwo{Data Formats}

\JWlthree{Voltage Data Point Files}
\label{sec:datapoint-files}

\JWlthree{Counter Files}
\label{sec:counter-files}

%-------------------------------------------------------------------------------
\JWltwo{Tools}

Since the building of a reasonable energy model is not an easy task, numerous
tools have been used. Most of the tools were developed specifically for the
purpose of this study thesis. All of them are open-sourced and available on
\JWnamedlink{https://github.com/weissi/studienarbeit}{GitHub}.

In this chapter, we will give an overview of the software used, both standard
software and tools developed specifically to be able to write this paper.

\JWlthree{Standard Software}

\begin{itemize}

\item \JWTprotobuf for saving and loading of all kinds of data

\item \JWnamedlink{http://www.r-project.org/}{R} for statistical computations

\item \JWnamedlink{http://cran.r-project.org/web/packages/leaps/}{leaps package}
      , a R library for regression subset selection (to minimize the set of
      available performance counters)

\item \JWnamedlink{http://stat.ethz.ch/R-manual/R-devel/library/stats/html/lm.html}
      {lm library}, a R library used to fit linear models

\item \JWnamedlink{http://kernel.org}{Linux Kernel}

\item \JWnamedlink{http://gnu.org}{numerous GNU tools}

\end{itemize}


\JWlthree{Special Developments}


\JWlfour{\JWTlibdp}

\JWTlibdp is responsible for loading and saving the measured data points from
and to the \JWTprotobuf files. It's API is straight forward and it is able to
handle very big files. The API can be found in
\JWpath{libdatapoints/datapoints.h}.


\JWlfour{\JWTfcw}

\JWTfcw can calculate the electrical work from data points files very quickly.
As explained in section \ref{sec:calc-work} it integrates to electrical power.
But since we obtain discrete data by sampling (see section
\ref{sec:measuring-setup}) the integration can be done quite fast:

\JWtodo{integral P = sum P * diff}

A typical call to \JWTfcw looks like

\begin{lstlisting}[style=Shell]
fastcalcwork captured-17:15:00.dpts CPU 0.01 TRIGGER
\end{lstlisting}

Using the command line above, \JWTfcw will calculate the electrical work with a
measuring resistor of \SI{0.01}{\ohm}. The measured data points will be taken
from column \texttt{CPU}, the analog trigger's (see section
\ref{sec:measuring-setup}) value from \texttt{TRIGGER}.


\JWlfour{\JWTdc}

\JWTdc is the tool for retrieving the performance counter's values on the target
machine. Giving it a set of performance counters and a command to execute, it
will record the counters while the command is running. It always works
system-wide and saves values by counter and by CPU enabled.

A working example:

\begin{lstlisting}[style=Shell]
dumpcounters -e CPU_CLK_UNHALTED,INST_RETIRED -r ls
\end{lstlisting}

The exact data format is documented in \JWpath{protos/perf-counters.proto}.

\JWlfour{\JWTde}

\JWTde exports data point files (see section \ref{sec:datapoint-files}) to a
format \JWTR can easily use.

\JWtodo{bib ref to R table format}.

\JWlfour{\JWTdd}

\JWlfour{\JWTcbs suite}

\JWlfour{\JWTbsle}

\JWlfour{high-level scripts}

% vim: set spell spelllang=en_us fileencoding=utf8 :
