\JWlone{Conclusion}

\begin{itemize}

\item What are the achievements? First Sandy Bridge model.

\item Praise me ;-)

\end{itemize}


% #  PROBLEMS  #################################################################
\JWltwo{Problems}
\label{sec:problems}

What are the problems, what has not been looked at?

\begin{itemize}

\item Suboptimal counter selection? (Multi-core: better events)

\item Short processes lead to fugly results?

\item No true multi-threading (processes running simultaneously do not share
resources

\item Big problem classes as in the following sub sections.

\end{itemize}


%-  no advances features  ------------------------------------------------------
\JWlthree{No DVFS, Hyper-Threading, ACPI Processor States, Turbo Mode, Linux
Dynamic Tics}
\label{sec:no-advances-features}

Because of the limited amount of time and the start from scratch of this thesis,
some enery-relevant processor and operating system features have been disabled.
Since all of these either should work with the current model or individually
raise some kind of events, this is not seen as a major drawback of this work.
Future work may certainly be able to flawlessly integrate them into the known
model. In particular the feature which lasted disabled were:

\begin{itemize}

\item Dynamic Frequency and Voltage Scaling \cite{wiki:DVFS}

\item Hyper-Threading \cite{wiki:HT}

\item ACPI Processor States other than C0 \cite{wiki:ACPI}

\item Intel\TReg{} Turbo Boost Technology 2.0 \cite{wiki:IntelTurboBoost}

\end{itemize}


%-  no auxiliary processing units  ---------------------------------------------
\JWlthree{No Auxiliary Processing Units}
\label{sec:no-aux-units}

In addition to the advanced processor and operating system features meantioned
in chapter \ref{sec:no-advanced-features}, all auxiliary processing units could
not be taken into account. The problem with the auxiliary processing units such
as the floating point unit (FPU), MMX\cite{wiki:MMX}, SSE\cite{wiki:SSE} and
AVX\cite{wiki:AVX} is that they are not or not well covered by the CPU's
performance events (see \cite{intel2011events}). So, they act somehow as a
black box not reveiling the work they do internally. That makes it virtually
impossible to count their energy with the current model.



% #  OUTLOOK  ##################################################################
\JWltwo{Outlook}
\label{sec:outlook}

What could be looked at in future? (what's missing?)

See \ref{sec:problems} :-(.
