\JWlone{Conclusion}

As of our knowledge the energy model presented in this thesis is the first
performance event counter based for the Intel\TReg{} Sandy Bridge
microarchitecture. More than that, it is the first for CPUs with more than two
cores. The evaluation results prove the general concept is also applicable to
today's and tomorrow's multi--core CPUs.

The software developed along with this thesis provides a convenient and freely
available way to build energy models in the future.


% #  PROBLEMS  #################################################################
\JWltwo{Problems and Outlook}
\label{sec:problems}

Even though the resulting energy model already proved its efficiency and
necessity, there is room for further improvements. On the one hand for the model
itself, on the other hand for the process of building energy models for new
target architectures. First, the restrictions mentioned in chapter
\ref{sec:restrictions} should be eliminated. To improve the practical usefulness
real multi--threading programs should be better kept in mind. The challenge with
threads is that shared memory regions get accessed on the same time. This will
probably attract interest on other or additional performance events which are
not well covered here. The latter also shows the need to develop a convenient
event selection process. Supplementary, the upcoming many--core architectures
with $\gg 4$ cores might become challenging.

% vim: set spell spelllang=en_us fileencoding=utf8 : syntax spell toplevel :
